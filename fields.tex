% !TeX TS-program = xelatex
% !TeX encoding = UTF-8
% !TeX spellcheck = en_US
% !TeX root = document.tex

\chapter{Field Theory}

\begin{dfnbox}{Field}
	A \dfntxt{field} is a commutative ring $\alg{K; 0, 1, -, +, \cdot}$ that satisfies the following additional requirements:
	\begin{dfnitems}
		\item \dfntxt{Nontriviality}: $0 \ne 1$.
		\item \dfntxt{Invertibility}: Every element of $K \setminus \{0\}$ has a (two-sided) inverse, i.e., $K^\times = K \setminus \{0\}$.
	\end{dfnitems}
\end{dfnbox}

\begin{thmbox}{Quotient by Maximal Ideal is a Field}
	Let $R$ be a commutative ring, and let $I \normalin R$ be an ideal. The quotient ring $R/I$ is a field if and only if $I$ is a maximal ideal.
\end{thmbox}

\begin{dfnbox}{Field of Fractions, $\Frac(R)$, Numerator, Denominator, $a/b$, $\frac{a}{b}$}
	Let $R$ be an integral domain. The \dfntxt{field of fractions} of $R$, denoted by $\Frac(R)$, is the set of equivalence classes of the relation $\sim$ defined on $R \times (R \setminus \{0\})$ as follows: $(a, b) \sim (c, d)$ if and only if $ad = bc$. We denote by $a/b$ or $\frac{a}{b}$ the equivalence class of the pair $(a, b) \in R \times (R \setminus \{0\})$.
\end{dfnbox}

For this definition to make sense, we need to verify that $\sim$ is an equivalence relation on $R \times (R \setminus \{0\})$. Reflexivity and symmetry are straightforward consequences of the commutativity of $R$. Indeed, let $a, c, e \in R$ and $b, d, f \in R \setminus \{0\}$ be arbitrary. Then
\[ ab = ba \implies (a, b) \sim (a, b) \]
shows that $\sim$ is reflexive, and
\[ (a, b) \sim (c, d) \implies ad = bc \implies cb = da \implies (c, d) \sim (a, b) \]
shows that $\sim$ is symmetric. Establishing transitivity requires us to invoke the lack of zero divisors in $R$. Suppose that $(a, b) \sim (c, d)$ and $(c, d) \sim (e, f)$; this means that $ad - bc = 0$ and $cf - de = 0$. It follows that
\[ 0 = (ad - bc)f + b(cf - de) = adf - bcf + bcf - bde = adf - bde = d(af - be). \]
Since $d \ne 0$ by hypothesis, we know that $d$ is not a zero divisor, and hence that $af - be = 0$. This proves that $(a, b) \sim (e, f)$, completing the proof that $\sim$ is transitive.

\begin{thmbox}{Field of Fractions is a Field}
	\textbf{Theorem:} If $\alg{R; 0_R, 1_R, -_R, +_R, \cdot_R}$ is an integral domain, then $\alg{\Frac(R); 0, 1, -, +, \cdot}$, is a field, where the distinguished elements $0, 1 \in \Frac(R)$, the unary operation $-: \Frac(R) \to \Frac(R)$, and the binary operations $+, \cdot: \Frac(R) \times \Frac(R) \to \Frac(R)$ are defined as follows:
	\[ 0 \coloneq \frac{0_R}{1_R}
	\qquad\qquad
	1 \coloneq \frac{1_R}{1_R} \]
	\[ -\frac{a}{b} \coloneq \frac{-_R a}{b}
	\qquad\qquad
	\frac{a}{b} + \frac{c}{d} \coloneq \frac{(a \cdot_R d) +_R (b \cdot_R c)}{b \cdot_R d}
	\qquad\qquad
	\frac{a}{b} \cdot \frac{c}{d} \coloneq \frac{a \cdot_R c}{b \cdot_R d} \]
\end{thmbox}

\begin{dfnbox}{Field Homomorphism}
	A \dfntxt{field homomorphism} is a ring homomorphism $f: K \to L$ where the domain $K$ and codomain $L$ are both fields.
\end{dfnbox}

Note that field homomorphisms are defined as \textit{ring} homomorphisms, not \textit{rng} homomorphisms, so they are required to map $1$ to $1$.

We now prove a crucial fact that markedly distinguishes field theory from group theory and ring theory.

\begin{thmbox}{Fields Have Two Ideals}
	\textbf{Theorem:} Let $K$ be a field. If $I \le K$ is a one-sided ideal, then either $I = \{0\}$ or $I = K$.
\tcblower
	\textit{Proof:} A one-sided ideal $I \le K$ must contain $0$ by definition. If $I$ contains any other element of $K$, then $I$ contains a unit, and hence contains every element of $K$.
\end{thmbox}

\begin{thmbox}{Every Field Homomorphism is a Monomorphism}
	\textbf{Theorem:} Every rng homomorphism $f: K \to R$ from a field $K$ to a rng $R$ is either injective or trivial (i.e., $f(x) = 0$ for all $x \in K$).
\tcblower
	\textit{Proof:} Either $\ker f = \{0\}$, in which case $f$ is injective, or $\ker f = K$, in which case $f$ is trivial.
\end{thmbox}

In field theory, it is conventional to regard a monomorphism $f: K \injto L$ as an \textit{embedding} of $K$ into $L$. Under this interpretation, the preceding result shows that the only possible relationship between two fields (via a field homomorphism) is for one to be contained inside the other. For this reason, field theory does not adopt the language of homomorphisms that permeates group theory and ring theory. Instead, field theory is written in the language of \textit{subfields} and \textit{field extensions}.

\begin{dfnbox}{Subfield}
	Let $K$ be a field. A \dfntxt{subfield} of $K$ is a subring $L \le K$ that satisfies the following additional requirement:
	\begin{dfnitems}
		\item \dfntxt{Closed under inversion}: If $x \in L \setminus \{0\}$, then $x^{-1} \in L$.
	\end{dfnitems}
\end{dfnbox}

Not every subring of a field is a subfield. For example, $\Z$ is a subring but not a subfield of $\Q$.

\begin{dfnbox}{Field Extension, $L/K$}
	Let $L$ be a field. If $K$ is a subfield of $L$, then we say that $L$ is an \dfntxt{extension} of $K$, and we say that $L/K$ (pronounced as ``$L$ \textit{over} $K$'') is a \dfntxt{field extension}.
\end{dfnbox}

Somewhat confusingly, the notation $L/K$ is not intended to represent a quotient of any kind. It is simply a strange (but historically traditional) notation for an ordered pair $(L, K)$ of fields, carrying the additional information that the second field is contained in the first. Field theory has no use for quotients, since fields have no nontrivial proper ideals.

\begin{dfnbox}{Degree, $[L : K]$, Finite Extension}
	The \dfntxt{degree} of a field extension $L/K$, denoted by $[L : K]$, is the dimension of $L$ as a vector space over $K$. If $[L : K]$ is finite, then we call $L/K$ a \dfntxt{finite extension}.
\end{dfnbox}

Note that the individual fields $K$ and $L$ involved in a finite extension $L/K$ are allowed to have infinite cardinality. The phrase ``finite extension'' specifies that the \textit{extension} is finite, not that the \textit{fields} are finite.

\begin{dfnbox}{Algebraic Element, Transcendental Element}
	Let $L/K$ be a field extension. We say that an element $\alpha \in L$ is \dfntxt{algebraic} over $K$ if there exists a polynomial $p \in K[x]$ such that $p(\alpha) = 0$. If no such polynomial exists, then we say that $\alpha$ is \dfntxt{transcendental} over $K$.
\end{dfnbox}

\begin{dfnbox}{Algebraic Extension, Transcendental Extension}
	An \dfntxt{algebraic extension} is a field extension $L/K$ in which every element of $L$ is algebraic over $K$. On the other hand, a \dfntxt{transcendental extension} is a field extension $L/K$ in which $L$ contains an element that is transcendental over $K$.
\end{dfnbox}

\begin{thmbox}{Minimal Polynomials Exist}
	\textbf{Theorem:} Let $L/K$ be a field extension. If $\alpha \in L$ is algebraic over $K$, then there exists a unique monic polynomial in $K[x]$ of minimal degree satisfying $p(\alpha) = 0$.
\tcblower
	\textit{Proof:} The existence of such a polynomial is clear, since by the hypothesis that $\alpha$ is algebraic over $K$, there exists a polynomial $p \in K[x]$ satisfying $p(\alpha) = 0$. (This polynomial can be made monic by dividing it by its leading coefficient.) Because the degrees of monic polynomials (i.e., natural numbers) are well-ordered, we can conclude that there exists a minimal-degree monic polynomial having this property.

	We verify uniqueness by contradiction. Suppose that there exist two distinct monic polynomials $p, q \in K[x]$ of minimal degree that satisfy $p(\alpha) = q(\alpha) = 0$. It follows that $p-q \in K[x]$ is a polynomial of strictly smaller degree that satisfies $(p-q)(\alpha) = p(\alpha) - q(\alpha) = 0 - 0 = 0$, contradicting the minimality of $p$ and $q$.
\end{thmbox}

\begin{dfnbox}{Minimal Polynomial}
	Let $L/K$ be a field extension, and let $\alpha \in L$ be algebraic over $K$. The \dfntxt{minimal polynomial} of $\alpha$ over $K$ is the unique monic polynomial $p \in K[x]$ of minimal degree satisfying $p(\alpha) = 0$.
\end{dfnbox}

\begin{thmbox}{Minimal Polynomials are Irreducible}
	\textbf{Theorem:} Let $L/K$ be a field extension. If $p \in K[x]$ is the minimal polynomial of an algebraic element $\alpha \in L$, then $p$ is irreducible in $K[x]$.
\tcblower
	\textit{Proof:} If $p$ could be written as the product of two non-constant polynomials $q, r \in K[x]$, then $0 = p(\alpha) = q(\alpha) r(\alpha)$ would imply that $q(\alpha) = 0$ or $r(\alpha) = 0$, contradicting the minimality of $p$.
\end{thmbox}

\begin{dfnbox}{Annihilator Ideal}
	Let $L/K$ be a field extension. The \dfntxt{annihilator ideal} of an element $\alpha \in L$ over $K$ is the subset of $K[x]$
\end{dfnbox}

\begin{thmbox}{Minimal Polynomial Divides Every Other Polynomial}
\end{thmbox}

\begin{dfnbox}{Finite Field}
	A \dfntxt{finite field} is a field whose underlying set has finite cardinality.
\end{dfnbox}

\begin{thmbox}{Finite Fields have Cyclic Multiplicative Groups}
	\textbf{Theorem:} If $K$ is a finite field, then $K^\times$ is a cyclic group.
\end{thmbox}

\begin{dfnbox}{Simple Extension, Primitive Element}
	A field extension $L/K$ is called a \dfntxt{simple extension} if there exists an element $\alpha \in L$ such that $L = K(\alpha)$, i.e., every element of $L$ can be expressed as a rational function of $\alpha$ with coefficients in $K$. Such an element $\alpha$ is called a \dfntxt{primitive element} of $L$ over $K$.
\end{dfnbox}

\begin{thmbox}{Primitive Element Theorem}
	\textbf{Theorem:} Let $K$ be a field. A finite extension $L/K$ is simple if and only if there exist finitely many intermediate subfields $F$ satisfying $K \le F \le L$.
\tcblower
	\textit{Proof:} We first consider the case of a finite base field $K$. In this case, the extension field $L$, being a finite extension of a finite field, is also finite. This implies that both sides of the desired ``if and only if'' statement are true. In particular, $L$ has finitely many subfields, and $L^\times$ is a cyclic group, so its generator is a primitive element.

	Now suppose that the base field $K$ is infinite. There are two implications that need to be proven.
	\begin{dfnitems}
		\item Suppose $L/K$ is a simple extension with primitive element $\alpha \in L$,
		\item Suppose there are finitely many intermediate subfields $F$ satisfying $K \le F \le L$.
	\end{dfnitems}
\end{thmbox}

\begin{dfnbox}{(Algebraic) Number Field}
	An \dfntxt{algebraic number field}, or simply a \dfntxt{number field}, is a finite extension of $\Q$.
\end{dfnbox}

\begin{dfnbox}{Ring of Integers}
	Let $K$ be an algebraic number field. An element $\alpha \in K$ is an \dfntxt{algebraic integer} if there exists a monic polynomial $p \in \Z[x]$ such that $p(\alpha) = 0$. The set of all algebraic integers in $K$ is called the \dfntxt{ring of integers} of $K$, denoted by $\symcal{O}_K$.
\end{dfnbox}
