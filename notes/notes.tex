% !TeX program = xelatex

\documentclass[12pt]{report}
\usepackage{dkzhang}
\usepackage[margin=1in]{geometry}
\usepackage{cancel}

\title{Notes on Mathematics}
\author{David K. Zhang}
\date{Last modified \today}

\begin{document}
\maketitle
\pagenumbering{roman}
\listofdefinition
\clearpage
\listoftheorem
\clearpage
\pagenumbering{arabic}



\chapter{Set Theory}

\begin{dfnbox}{Empty Set, $\emptyset$}
	The \dfntxt{empty set}, denoted by $\emptyset$, is the set that contains no elements at all.
\end{dfnbox}

In other words, $x \notin \emptyset$ for all $x$.

\begin{dfnbox}{Non-empty Set}
	A set is \dfntxt{non-empty} if it contains at least one element.
\end{dfnbox}

\begin{dfnbox}{Subset, $A \subseteq B$}
	Let $A$ and $B$ be sets. We say that $A$ is a \dfntxt{subset} of $B$, denoted by $A \subseteq B$, if every element of $A$ is also an element of $B$, i.e., for all $x \in A$ we have $x \in B$.
\end{dfnbox}

\begin{dfnbox}{Superset, $A \supseteq B$}
	Let $A$ and $B$ be sets. We say that $A$ is a \dfntxt{superset} of $B$, denoted by $A \supseteq B$, if $B$ is a subset of $A$.
\end{dfnbox}

Note that the empty set is a subset of every set, and that every set is a superset of the empty set.

\begin{dfnbox}{Set Equality, $A = B$, $A \ne B$}
	Let $A$ and $B$ be sets. We say that $A$ and $B$ are \dfntxt{equal}, denoted by $A = B$, if $A \subseteq B$ and $B \subseteq A$. If this is not the case, then we write $A \ne B$ to denote that $A$ and $B$ are not equal.
\end{dfnbox}

Using this notation, we write $A \ne \emptyset$ to denote that a set $A$ is non-empty.

\begin{dfnbox}{Proper Subset, $A \subsetneq B$}
	Let $A$ and $B$ be sets. We say that $A$ is a \dfntxt{proper subset} of $B$, denoted by $\subsetneq$, if $A \subseteq B$ and $A \ne B$.
\end{dfnbox}

\begin{dfnbox}{Proper Superset, $A \supsetneq B$}
	Let $A$ and $B$ be sets. We say that $A$ is a \dfntxt{proper superset} of $B$, denoted by $\supsetneq$, if $A \supseteq B$ and $A \ne B$.
\end{dfnbox}

\begin{dfnbox}{Power Set, $\powerset(A)$, $2^A$}
	Let $A$ be a set. The \dfntxt{power set} of $A$, denoted by $\powerset(A)$ or $2^A$, is the set containing all subsets of $A$.
\end{dfnbox}

Note that $\powerset(A)$ contains both the empty set $\emptyset$ and the set $A$ itself.

\begin{dfnbox}{Union, $A \cup B$}
	Let $A$ and $B$ be sets. The \dfntxt{union} of $A$ and $B$, denoted by $A \cup B$, is the set containing all elements of $A$ together with all elements of $B$.
	\[ x \in (A \cup B) \iff x \in A \text{ or } x \in B \]
\end{dfnbox}

\begin{dfnbox}{Intersection, Intersects, $A \cap B$}
	Let $A$ and $B$ be sets. The \dfntxt{intersection} of $A$ and $B$, denoted by $A \cap B$, is the set containing all elements that are present in both $A$ and $B$.
	\[ x \in (A \cap B) \iff x \in A \text{ and } x \in B \]
	We say that $A$ \dfntxt{intersects} $B$ if the intersection $A \cap B$ is non-empty.
\end{dfnbox}

\begin{dfnbox}{Difference, Relative Complement, $A \setminus B$}
	Let $A$ and $B$ be sets. The \dfntxt{difference} of $A$ and $B$, also known as the \dfntxt{relative complement} of $B$ in $A$, denoted by $A \setminus B$, is the set containing all the elements of $A$ which are not elements of $B$.
	\[ A \setminus B \coloneq \{ x \in A : x \notin B \} \]
\end{dfnbox}

\begin{dfnbox}{Cartesian Product, $A \times B$}
	Let $A$ and $B$ be sets. The \dfntxt{Cartesian product} of $A$ and $B$, denoted by $A \times B$, is the set of all ordered pairs $(a, b)$ consisting of an element $a \in A$ followed by an element $b \in B$.
	\[ A \times B \coloneq \{ (a, b) : a \in A \text{ and } b \in B \} \]
\end{dfnbox}

\begin{dfnbox}{Relation, Binary Relation, Relates, $a \mathrel{R} b$, $a \mathrel{\cancel{R}} b$}
	Let $A$ and $B$ be sets. A \dfntxt{relation} between $A$ and $B$, also known as a \dfntxt{binary relation} between $A$ and $B$, is a subset $R \subseteq A \times B$ of their Cartesian product.

	Given $a \in A$ and $b \in B$, we write $a \mathrel{R} b$ to denote that  $(a, b) \in R$. If this is the case, then we say that $R$ \dfntxt{relates} $a$ to $b$. Similarly, we write $a \mathrel{\cancel{R}} b$ to denote that $(a, b) \notin R$.

	If $A = B$, then instead of saying that $R$ is a relation \textit{between} $A$ and $A$, we simply say that $R$ is a binary relation \textit{on} $A$.
\end{dfnbox}

For example, the less-than relation $<$ is a binary relation on the set $\N$ of natural numbers. We write $3 < 4$ to indicate that the ordered pair $(3, 4) \in \N \times \N$ is an element of the relation $<$.

\begin{dfnbox}{Function, Maps, Domain, Codomain, $f: A \to B$, $f(x)$}
	Let $A$ and $B$ be sets. A \dfntxt{function} from $A$ to $B$ is a binary relation $f \subseteq A \times B$ between $A$ and $B$ that has the following property: for each $x \in A$, there exists a unique $y \in B$ such that $x \mathrel{f} y$. Instead of the word ``relates,'' we say that the function $f$ \dfntxt{maps} $x$ to $y$. We call the set $A$ the \dfntxt{domain} of $f$, and we call $B$ the \dfntxt{codomain} of $f$.

	We write $f: A \to B$ to denote that $f$ is a function from $A$ to $B$, and given $x \in A$, we write $f(x)$ to denote the unique element of $B$ that is related to $x$ by $f$.
\end{dfnbox}

\begin{dfnbox}{Injective Function, One-to-one Function, Injection, $f: A \injto B$}
	Let $A$ and $B$ be sets. A function $f: A \to B$ is \dfntxt{injective}, or \dfntxt{one-to-one}, or an \dfntxt{injection}, if no two distinct elements of $A$ are mapped to the same element of $B$ by $f$.
	\[ \forall x, y \in A : x \ne y \implies f(x) \ne f(y) \]
	We write $f: A \injto B$ to denote that $f$ is an injective function from $A$ to $B$.
\end{dfnbox}

\begin{dfnbox}{Surjective Function, Onto Function, Surjection, $f: A \surjto B$}
	Let $A$ and $B$ be sets. A function $f: A \to B$ is \dfntxt{surjective}, or \dfntxt{onto}, or a \dfntxt{surjection}, if every element of $B$ is mapped to by $f$.
	\[ \forall y \in B \ \exists x \in A : f(x) = y \]
	We write $f: A \surjto B$ to denote that $f$ is a surjective function from $A$ to $B$.
\end{dfnbox}

\begin{dfnbox}{Bijective Function, One-to-one Correspondence, Bijection, $f: A \bijto B$}
	Let $A$ and $B$ be sets. A function $f: A \to B$ is \dfntxt{bijective}, or a \dfntxt{one-to-one correspondence}, or a \dfntxt{bijection}, if $f$ is both injective and surjective. We write $f: A \bijto B$ to denote that $f$ is a bijective function from $A$ to $B$.
\end{dfnbox}

\begin{dfnbox}{Image, $f[S]$}
	Let $A$ and $B$ be sets. Given a function $f: A \to B$, the \dfntxt{image} of a subset $S \subseteq A$ under $f$, denoted by $f[S]$, is the set defined by
	\[ f[S] \coloneq \{ f(x) : x \in S \}. \]
\end{dfnbox}

In other words, $f[S]$ is the subset of $B$ containing all elements that $f$ maps to from $S$. For example, consider the function $f: \R \to \R$ defined by $f(x) \coloneq x^2$. The image of the interval $(-1, 1)$ under $f$ is the interval $f[(-1, 1)] = [0, 1)$.

\begin{dfnbox}{Inverse Image, Preimage, $f^{-1}[S]$}
	Let $A$ and $B$ be sets. Given a function $f: A \to B$, the \dfntxt{inverse image} or \dfntxt{preimage} of a subset $S \subseteq B$ under $f$, denoted by $f^{-1}[S]$, is the set defined by
	\[ f^{-1}[S] \coloneq \{ x \in A : f(x) \in S \}. \]
\end{dfnbox}

In other words, $f^{-1}[S]$ is the subset of $A$ containing all elements that map into $S$ under $f$. For example, consider the function $f: \R \to \R$ defined by $f(x) \coloneq x^2$. The inverse image of the interval $(1, 4)$ is the union of the two intervals $f^{-1}[(1, 4)] = (-2, -1) \cup (1, 2)$.

\begin{dfnbox}{Range}
	Let $A$ and $B$ be sets. The \dfntxt{range} of a function $f: A \to B$ is the image $f[A]$ of $A$ under $f$.
\end{dfnbox}

Note that a function is surjective if and only if its range equals its codomain.

\begin{dfnbox}{Identity Function, $\id_A$}
	Let $A$ be a set. The \dfntxt{identity function} on $A$ is the function $\id_A: A \to A$ defined by $\id_A(a) \coloneq a$ for all $a \in A$.
\end{dfnbox}

\begin{dfnbox}{Cartesian Product, $\prod_{i \in I} A_i$}
	Let $I$ be an index set, and let $\{ A_i \}_{i \in I}$ be an indexed family of sets. The \dfntxt{Cartesian product} of $\{ A_i \}_{i \in I}$, denoted by $\prod_{i \in I} A_i$, is the set of all functions $f: I \to \bigcup_{i \in I} A_i$ such that $f(i) \in A_i$ for all $i \in I$.
	\[ \prod_{i \in I} A_i \coloneq \left\{ f: I \to \bigcup_{i \in I} A_i \ \middle|\ \forall i \in I : f(i) \in A_i \right\} \]
\end{dfnbox}



\chapter{Topology}

\begin{dfnbox}{Topology}
	Let $X$ be a set. A \dfntxt{topology} on $X$ is a collection $T \subseteq \powerset(X)$ of subsets of $X$ that satisfies the following requirements:
	\begin{boxitems}
		\item \dfntxt{Empty set is open}: $\emptyset \in T$.
		\item \dfntxt{Whole set is open}: $X \in T$.
		\item \dfntxt{Closed under arbitrary unions}: If $S \subseteq T$, then $\bigcup S \in T$.
		\item \dfntxt{Closed under finite intersections}: If $S \subseteq T$ and $\abs{S}$ is finite, then $\bigcap S \in T$.
	\end{boxitems}
\end{dfnbox}

\begin{dfnbox}{Topological Space, Point, Open Set}
	A \dfntxt{topological space} is an ordered pair $(X, T)$ consisting of a set $X$, called the \dfntxt{underlying set}, and a topology $T$ on $X$. The elements of $X$ are called the \dfntxt{points} of the topological space $(T, X)$, and the elements of $T$ are called \dfntxt{open sets}.
\end{dfnbox}

\begin{dfnbox}{Closed Set}
	Let $(X, T)$ be a topological space. A subset $F \subseteq X$ is \dfntxt{closed} if $X \setminus F \in T$.
\end{dfnbox}

\begin{dfnbox}{Clopen Set}
	Let $(X, T)$ be a topological space. A subset of $X$ is \dfntxt{clopen} if it is both open and closed.
\end{dfnbox}

\begin{dfnbox}{Open Neighborhood}
	Let $(X, T)$ be a topological space. An \dfntxt{open neighborhood} of a point $x \in X$ is an open set $U \in T$ that contains $x$.
\end{dfnbox}

\begin{dfnbox}{Neighborhood}
	Let $(X, T)$ be a topological space, and let $x \in X$. A subset $N \subseteq X$ is a \dfntxt{neighborhood} of $x$ if there exists an open set $U \in T$ such that $x \in U$ and $U \subseteq N$.
\end{dfnbox}

\begin{dfnbox}{Limit Point, Cluster Point, Accumulation Point}
	Let $(X, T)$ be a topological space, and let $A \subseteq X$. A point $x \in X$ is a \dfntxt{limit point} of $A$, also known as a \dfntxt{cluster point} or \dfntxt{accumulation point}, if every open neighborhood of $x$ contains a point in $A$ other than $x$ itself.
\end{dfnbox}

In other words, $x$ is a limit point of $A$ if every open neighborhood of $x$ intersects $A \setminus \{x\}$.

\begin{dfnbox}{Derived Set, $A'$}
	Let $(X, T)$ be a topological space, and let $A \subseteq X$. The \dfntxt{derived set} of $A$, denoted by $A'$, is the set of all limit points of $A$ in $X$.
	\[ A' \coloneq \{ x \in X : x \text{ is a limit point of } A \} \]
\end{dfnbox}

\begin{thmbox}{Closed $\iff$ Contains All Limit Points}
	\textbf{Theorem:} Let $(X, T)$ be a topological space. A subset $A \subseteq X$ is closed if and only if $A$ contains all of its limit points, i.e., $A' \subseteq A$.
\tcblower
	\textit{Proof:} Suppose $A \subseteq X$ is closed, and let $x \in A'$ be a limit point of $A$. We will prove that $x \in A$ by contradiction. If $x \notin A$, then $x \in X \setminus A$. By hypothesis, $A$ is closed, so $X \setminus A$ is open. It follows that $X \setminus A$ is an open neighborhood of $x$ that does not intersect $A$, contradicting the hypothesis that $x$ is a limit point of $A$.
	
	Conversely, suppose that $A' \subseteq A$. We must prove that $A$ is closed, i.e., that $X \setminus A$ is open. For each point $x \in X \setminus A$, the hypothesis $A' \subseteq A$ implies that $x$ is not a limit point of $A$. Hence, by definition, there exists an open neighborhood $U_x \in T$ of $x$ that does not intersect $A$. The union of all such neighborhoods is precisely $X \setminus A = \bigcup_{x \in X \setminus A} U_x$, which, being a union of open sets, is consequently open.
\end{thmbox}

\begin{dfnbox}{Interior, $A^\circ$}
	Let $(X, T)$ be a topological space, and let $A \subseteq X$. The \dfntxt{interior} of $A$, denoted by $A^\circ$, is the union of all open sets $U \in T$ that are subsets of $A$.
	\[ A^\circ \coloneq \bigcup \ \{ U \in T : U \subseteq A \} \]
\end{dfnbox}

\begin{dfnbox}{Closure, $\overline{A}$}
	Let $(X, T)$ be a topological space, and let $A \subseteq X$. The \dfntxt{closure} of $A$, denoted by $\overline{A}$, is the intersection of all closed sets $F \subseteq X$ that are supersets of $A$.
	\[ \overline{A} \coloneq \bigcap \ \{ F \subseteq X : X \setminus F \in T \text{ and } F \supseteq A \} \]
\end{dfnbox}

\begin{dfnbox}{Continuous Function}
	Let $(X, T_X)$ and $(Y, T_Y)$ be topological spaces. A function $f: X \to Y$ is \dfntxt{continuous} with respect to the topologies $T_X$ and $T_Y$ if, for every $U \in T_Y$, we have $f^{-1}[U] \in T_X$.
\end{dfnbox}

\begin{dfnbox}{Metric, Distance Function}
	Let $X$ be a set. A \dfntxt{metric} on $X$, also known as a \dfntxt{distance function} on $X$, is a function $d: X \times X \to \R$ that satisfies the following requirements:
	\begin{boxitems}
		\item \dfntxt{Positive-definiteness}: $d(x, y) \ge 0$ for all $x, y \in X$, and $d(x, y) = 0$ if and only if $x = y$.
		\item \dfntxt{Symmetry}: $d(x, y) = d(y, x)$ for all $x, y \in X$.
		\item \dfntxt{Triangle inequality}: $d(x, z) \le d(x, y) + d(y, z)$ for all $x, y, z \in X$.
	\end{boxitems}
\end{dfnbox}

\begin{dfnbox}{Metric Space, Underlying Set, Point}
	A \dfntxt{metric space} is an ordered pair $(X, d)$ consisting of a set $X$, called the \dfntxt{underlying set} of the metric space, and a metric $d$ on $X$. The elements of $X$ are called the \dfntxt{points} of the metric space $(X, d)$.
\end{dfnbox}

\begin{dfnbox}{Continuity}
	Let $(X, d_X)$ and $(Y, d_Y)$ be metric spaces. A function $f: X \to Y$ is \dfntxt{continuous} at a point $x_0 \in X$ if, for every $\epsilon > 0$, there exists $\delta > 0$, such that for all $x \in X$, if $d_X(x, x_0) < \delta$, then $d_Y(f(x), f(x_0)) < \epsilon$.
\end{dfnbox}



\chapter{Group Theory}

\begin{dfnbox}{Group}
	A \dfntxt{group} is an algebraic structure $\alg{G; 1, {}^{-1}, \cdot}$ consisting of:
	\begin{boxitems}
		\item a set $G$, called the \dfntxt{underlying set};
		\item a distinguished element $1 \in G$, called the \dfntxt{identity element};
		\item a unary operation ${}^{-1}: G \to G$, written as $x \mapsto x^{-1}$, called \dfntxt{inversion};
		\item a binary operation $\cdot: G \times G \to G$, written as $(x, y) \mapsto x \cdot y$, called the \dfntxt{group operation} or \dfntxt{group product};
	\end{boxitems}
	satisfying the following requirements:
	\begin{boxitems}
		\item \dfntxt{Associative property}: $(x \cdot y) \cdot z = x \cdot (y \cdot z)$ for all $x, y, z \in G$.
		\item \dfntxt{Identity property}: $1 \cdot x = x \cdot 1 = x$ for all $x \in G$.
		\item \dfntxt{Inverse property}: $x \cdot x^{-1} = x^{-1} \cdot x = 1$ for all $x \in G$.
	\end{boxitems}
\end{dfnbox}

\begin{thmbox}{Cancellation Laws}
	\textbf{Theorem:} Let $\alg{G; 1, {}^{-1}, \cdot}$ be a group, and let $x, y, z \in G$.
	\begin{boxitems}
		\item \dfntxt{Left cancellation law}: If $x \cdot y = x \cdot z$, then $y = z$.
		\item \dfntxt{Right cancellation law}: If $x \cdot z = y \cdot z$, then $x = y$.
	\end{boxitems}
\tcblower
	\textit{Proof:} If $x \cdot y = x \cdot z$, then:
	\begin{align*}
		y & = 1 \cdot y                &  & \text{(identity property)}    \\
		  & = (x^{-1} \cdot x) \cdot y &  & \text{(inverse property)}     \\
		  & = x^{-1} \cdot (x \cdot y) &  & \text{(associative property)} \\
		  & = x^{-1} \cdot (x \cdot z) &  & \text{(by hypothesis)}        \\
		  & = (x^{-1} \cdot x) \cdot z &  & \text{(associative property)} \\
		  & = 1 \cdot z                &  & \text{(inverse property)}     \\
		  & = z                        &  & \text{(identity property)}
	\end{align*}
	Similarly, if $x \cdot z = y \cdot z$, then
	\[ x = x \cdot 1 = x \cdot (z \cdot z^{-1}) = (x \cdot z) \cdot z^{-1} = (y \cdot z) \cdot z^{-1} = y \cdot (z \cdot z^{-1}) = y \cdot 1 = y. \]
\end{thmbox}

\begin{dfnbox}{Abelian Group}
	An \dfntxt{abelian group} is a group $\alg{G; 1, {}^{-1}, \cdot}$ that satisfies the following additional requirement:
	\begin{boxitems}
		\item \dfntxt{Commutative property}: $x \cdot y = y \cdot x$ for all $x, y \in G$.
	\end{boxitems}
\end{dfnbox}



\chapter{Ring Theory}

In this chapter, we introduce a new class of algebraic structures, called rngs and rings, whose study is collectively called \dfntxt{ring theory}. Rngs and rings are more complicated than groups because their definition involves not one, but two binary operations.

\begin{dfnbox}{Rng}
	A \dfntxt{rng} (pronounced as ``\textit{rung}'') is an algebraic structure $\alg{R; 0, -, +, \cdot}$ consisting of:
	\begin{boxitems}
		\item a set $R$, called the \dfntxt{underlying set};
		\item a distinguished element $0 \in R$, called the \dfntxt{zero element};
		\item a unary operation $-: R \to R$, written as $x \mapsto -x$, called \dfntxt{negation};
		\item a binary operation $+: R \times R \to R$, written as $(x, y) \mapsto x + y$, called \dfntxt{addition};
		\item a binary operation $\cdot: R \times R \to R$, written as $(x, y) \mapsto x \cdot y$, called \dfntxt{multiplication};
	\end{boxitems}
	satisfying the following requirements:
	\begin{boxitems}
		\item \dfntxt{Additive structure}: $\alg{R; 0, -, +}$ is an abelian group.
		\item \dfntxt{Associativity}: $(x \cdot y) \cdot z = x \cdot (y \cdot z)$ for all $x, y, z \in R$.
		\item \dfntxt{Left distributivity}: $x \cdot (y + z) = (x \cdot y) + (x \cdot z)$ for all $x, y, z \in R$.
		\item \dfntxt{Right distributivity}: $(x + y) \cdot z = (x \cdot z) + (y \cdot z)$ for all $x, y, z \in R$.
	\end{boxitems}
\end{dfnbox}

The key ingredient in the definition of a rng is \dfntxt{distributivity}, which establishes a link between two different binary operations. We begin our study of rngs by proving a simple (but important) result to demonstrate the utility of the distributive property.

\begin{thmbox}{Multiplying by Zero Yields Zero}
	\textbf{Theorem:} Let $\alg{R; 0, -, +, \cdot}$ be a rng. For any $x \in R$, we have $0 \cdot x = x \cdot 0 = 0$.
\tcblower
	\textit{Proof:} Let $x \in R$ be given. Because $0$ is the identity element of the abelian group $\langle R; 0, -, + \rangle$, we have $0 = 0 + 0$. Using left distributivity, it follows that $0 \cdot x = (0 + 0) \cdot x = (0 \cdot x) + (0 \cdot x)$, and by canceling one copy of $0 \cdot x$ on both sides, we conclude that $0 \cdot x = 0$. We similarly apply right distributivity to the expression $x \cdot 0 = x \cdot (0 + 0) = (x \cdot 0) + (x \cdot 0)$ to conclude that $x \cdot 0 = 0$.
\end{thmbox}

\begin{dfnbox}{Zero Rng, Trivial Rng}
	The \dfntxt{zero rng} or \dfntxt{trivial rng} is the rng $\alg{ \{0\}; 0, -, +, \cdot }$ whose underlying set is a singleton containing the distinguished element $0$, and whose operations are defined by $-0 = 0 + 0 = 0 \cdot 0 = 0$.
\end{dfnbox}

\begin{dfnbox}{Nonzero Rng, Nontrivial Rng}
	A rng is \dfntxt{nonzero} or \dfntxt{nontrivial} if its underlying set contains more than one element.
\end{dfnbox}

\begin{dfnbox}{Subrng, $S \le R$}
	Let $\alg{R; 0, -, +, \cdot}$ be a rng. A \dfntxt{subrng} of $\alg{R; 0, -, +, \cdot}$ is a subgroup $S \le \alg{R; 0, -, +}$ that satisfies the following additional requirement:
	\begin{boxitems}
		\item \dfntxt{Closed under products}: If $x, y \in S$, then $x \cdot y$ in $S$.
	\end{boxitems}
	We write $S \le R$ to indicate that $S$ is a subrng of $\alg{R; 0, -, +, \cdot}$.
\end{dfnbox}

The zero rng is a subrng of every rng.

\begin{dfnbox}{Rng Homomorphism}
	Let $\alg{R; 0_R, -_R +_R, \cdot_R}$ and $\alg{S; 0_S, -_S +_S, \cdot_S}$ be rngs. A \dfntxt{rng homomorphism} from $R$ to $S$ is a function $f: R \to S$ that satisfies the following requirements:
	\begin{boxitems}
		\item \dfntxt{Preserves additive structure}: $f$ is a group homomorphism between the abelian groups $\alg{R; 0_R, -_R +_R}$ and $\alg{S; 0_S, -_S +_S}$.
		\item \dfntxt{Preserves products}: $f(x \cdot_R y) = f(x) \cdot_S f(y)$ for all $x, y \in R$.
	\end{boxitems}
\end{dfnbox}

Every rng homomorphism is also a group homomorphism, so the theory and terminology of group homomorphisms is immediately applicable to rng homomorphisms. For example, the kernel of a rng homomorphism $f: R \to S$ is still defined to be the set
\[ \ker f \coloneq \{ x \in R : f(x) = 0_S \}. \]

As with groups, we will often refer to a rng $\alg{R; 0, -, +, \cdot}$ by the name of its underlying set $R$ and denote the multiplication operation $\cdot$ by juxtaposition. We will never denote addition, subtraction, or negation by juxtaposition, so the symbols $+$ and $-$ will always be used.

\begin{dfnbox}{Ideal, Left Ideal, Right Ideal, One-Sided Ideal, Two-Sided Ideal, $I \normalin R$}
	Let $\alg{R; 0, 1, -, +, \cdot}$ be a rng.
	\begin{boxitems}
		\item A \dfntxt{left ideal} of $R$ is a subrng $I \le R$ that satisfies the following additional requirement:
		\begin{boxitems}
			\item \dfntxt{Absorbs left multiplication}: $rx \in I$ for all $r \in R$ and $x \in I$.
		\end{boxitems}
		\item A \dfntxt{right ideal} of $R$ is a subrng $I \le R$ that satisfies the following additional requirement:
		\begin{boxitems}
			\item \dfntxt{Absorbs right multiplication}: $xr \in I$ for all $x \in I$ and $r \in R$.
		\end{boxitems}
		\item A \dfntxt{one-sided ideal} of $R$ is a subrng $I \le R$ that is a left ideal or a right ideal (possibly both).
		\item A \dfntxt{two-sided ideal} of $R$, or simply an \dfntxt{ideal} of $R$, is a subrng $I \le R$ that is both a left ideal and a right ideal.
	\end{boxitems}
	We write $I \normalin R$ to denote that $I$ is a (two-sided) ideal of $R$.
\end{dfnbox}

\begin{dfnbox}{Rng of Formal Power Series, $R \lBrack x \rBrack$}
	Let $\alg{R; 0, -, +, \cdot}$ be a rng. The \dfntxt{rng of formal power series} over $\alg{R; 0, -, +, \cdot}$, denoted by $R \lBrack x \rBrack$, \todo{FINISH THIS DEFINITION}
\end{dfnbox}

\begin{dfnbox}{Rng of Polynomials, $R[x]$}
	Let $\alg{R; 0, -, +, \cdot}$ be a rng. The \dfntxt{rng of polynomials} over $\alg{R; 0, -, +, \cdot}$, denoted by $R[x]$, is the subrng of $R \lBrack x \rBrack$ \todo{FINISH THIS DEFINITION}
\end{dfnbox}

\begin{dfnbox}{Commutative Rng}
	A \dfntxt{commutative rng} is a rng $\alg{R; 0, -, +, \cdot}$ that satisfies the following additional requirement:
	\begin{boxitems}
		\item \dfntxt{Commutativity}: $x \cdot y = y \cdot x$ for all $x, y \in R$.
	\end{boxitems}
\end{dfnbox}

The word ``commutative'' in the term ``commutative rng'' emphasizes that the \textit{multiplication} operation is commutative. By definition, the addition operation is always commutative in every rng.

\begin{dfnbox}{Ring}
	A \dfntxt{ring} is an algebraic structure $\alg{R; 0, 1, -, +, \cdot}$ consisting of:
	\begin{boxitems}
		\item a set $R$, called the \dfntxt{underlying set};
		\item a distinguished element $0 \in R$, called the \dfntxt{zero element};
		\item a distinguished element $1 \in R$, called the \dfntxt{identity element};
		\item a unary operation $-: R \to R$, written as $x \mapsto -x$, called \dfntxt{negation};
		\item a binary operation $+: R \times R \to R$, written as $(x, y) \mapsto x + y$, called \dfntxt{addition};
		\item a binary operation $\cdot: R \times R \to R$, written as $(x, y) \mapsto x \cdot y$, called \dfntxt{multiplication};
	\end{boxitems}
	satisfying the following requirements:
	\begin{boxitems}
		\item \dfntxt{Rng structure}: $\alg{R; 0, -, +, \cdot}$ is a rng.
		\item \dfntxt{Identity}: $1 \cdot x = x \cdot 1 = x$ for all $x \in R$.
	\end{boxitems}
\end{dfnbox}

\begin{dfnbox}{Subring}
	Let $\alg{R; 0, 1, -, +, \cdot}$ be a ring. A \dfntxt{subring} of $\alg{R; 0, 1, -, +, \cdot}$ is a subrng $S \le \alg{R; 0, -, +, \cdot}$ that satisfies the following additional requirement:
	\begin{boxitems}
		\item \dfntxt{Contains the identity}: $1 \in S$.
	\end{boxitems}
\end{dfnbox}

In these notes, we will only use the notation $S \le R$ for \textit{subrngs}, not \textit{subrings}. This ensures consistency with the notation $I \normalin R$ for ideals (i.e., $I \normalin R \implies I \le R$), since an ideal is always a subrng, but not necessarily a subring.

\begin{dfnbox}{Field}
	A \dfntxt{field} is a commutative ring $\alg{K; 0, 1, -, +, \cdot}$ that satisfies the following additional requirements:
	\begin{boxitems}
		\item \dfntxt{Nontriviality}: $0 \ne 1$.
		\item \dfntxt{Invertibility}: Every element of $K \setminus \{0\}$ has a two-sided inverse, i.e., $K^\times = K \setminus \{0\}$.
	\end{boxitems}
\end{dfnbox}



\end{document}